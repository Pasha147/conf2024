\documentclass[a4paper,10pt]{article}
\usepackage[utf8]{inputenc}
\usepackage{fontspec}
\usepackage{xltxtra}
\usepackage[ukrainian]{babel}
\usepackage{xecyr}
\defaultfontfeatures{Scale=MatchLowercase,Mapping=tex-text}
\setmainfont{Times New Roman}
\usepackage{amssymb,amsfonts,amsmath,mathtext,esint}
\usepackage[a4paper, text={131mm, 225.5mm}]{geometry}
\usepackage[labelsep=period]{caption}
\usepackage{anyfontsize}
\usepackage{duckuments}
\usepackage{graphicx}

\begin{document}
\pagestyle{empty}
\begin{center}
\textbf{О.О. Гончар$^1$, І.М. Гончаренко$^{1,2}$ (по центру)}
\bigskip

\textbf{НАЗВА ВЕЛИКИМИ ЛІТЕРАМИ (ДО 12 СЛІВ)}
\bigskip

\textit{\fontsize{9}{9}\selectfont$^1$Інститут механіки ім. С.П.Тимошенка НАНУ,
вул. П.Нестерова, 3, 03057, Київ, Україна;
e-mail: email@inmech.kyiv.ua}

\textit{\fontsize{9}{9}\selectfont$^2$Національний технічний університет України
«Київський політехнічний інститут ім. Ігоря Сікорського»,
просп. Берестейський, 37, 03056, Київ, Україна}                                        \end{center}


Обсяг анотації 1-3 повні сторінки, набрані через один інтервал в текстовому редакторі.

\begin{equation}
 \Omega=\sqrt{\Xi-\dfrac{E^2}{Z}}
\end{equation}


\begin{figure}[ht]
\centering
\includegraphics[width=3cm]{example-image-duck}
\caption{}\label{example}
\end{figure}

Для растрових рисунків слід використовувати роздільну здатність не менше 300 точок на дюйм (dpi). Всі пояснення до рисунків слід вносити в текст.
\bigskip

{\fontsize{8.5}{8.5}\selectfont
КЛЮЧОВІ СЛОВА: ключові слова. (не більше 2-х рядків, 8,5 pt)}
\bigskip

Список літератури подається в кінці тез в алфавітному порядку. Джерела, опубліковані латиницею, наводяться за алфавітом після джерел, виданих кирилицею. Посилання на цитовані джерела повинні бути вказані в тексті у квадратних дужках (наприклад, [2, 11 – 13]). Приклади представлення основних типів джерел наведені нижче.
{\fontsize{8.5}{8.5}\selectfont
\begin{enumerate}
 \item Guz A.N. For the 100th Anniversary of the S. P. Timoshenko Institute of Mechanics of the National Academy of Sciences of Ukraine (NASU) // Int. Appl. Mech. – 2018. – 54, N 1. – P. 3 – 33.
 \item Guz I.A., Rodger A.A., Guz A.N., Rushchitsky J.J. Predicting the properties of nano\-composites with brush-like reinforcement // Carbon Nano Tubes New Engineering Tech\-no\-logies: Abstracts of the Int. Conf. CNTNET 07, University of Cambridge, Trinity College, United Kingdom. (Cambridge, 10-12 September 2007). – Cambridge, 2007. – P. 29.
 \item Timoshenko S.P., Gere J.M. Mechanics of Materials. – New York: Van Nostrand Reinhold Company, 1972. – 670 p.
\end{enumerate}
}

\begin{center}
\textbf{O.O. Gonchar$^1$, I.M. Goncharenko$^{1,2}$}

\textbf{TITLE IN CAPITAL LETTERS (CENTERED)}

\textit{\fontsize{9}{9}\selectfont$^1$S.P. Timoshenko Institute of Mechanics of the National Academy of Sciences, P. Nesterov Str., 3, 03057, Kyiv, Ukraine;
e-mail: email@inmech.kyiv.ua (centered)}

\textit{\fontsize{9}{9}\selectfont
$^2$National Technical University of Ukraine ''Igor Sikorsky Kyiv Polytechnic Institute'', Beresteyskyi Ave. 37, 03056, Kyiv, Ukraine}
\end{center}

The text of English annotation. The annotation should not exceed 5 lines.
\end{document}
